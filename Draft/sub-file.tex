\documentclass[example.tex]{subfiles}
%% NO NEED TO INPUT PREAMBLES HERE
%% packages are inherited; you can compile this file on its own
%% 

\onlyinsubfile{
\title{Subfile}
}

\begin{document}
\onlyinsubfile{
  \maketitle
  \thispagestyle{empty}
  %%%% Table of Contents
  \tableofcontents
}


\subsection{sub-section from a subfile}

This is content from a subfile that is self contained. 
It inherits all packages from the ``Master'' file, but can be compiled
on its own.  This is especially nice when the document has lots of
figures and you are only working on one section/subsection of the
document. \\


You can compile this subfile by using 
\verb+xelatex sub-file.tex+\\


The idea is that a person would copy and rename the file
and include their content.  When working on github with more than
one author this can be very helpfull, each person gets their own
subfile.  %  
\href{https://en.wikibooks.org/wiki/LaTeX/Modular_Documents#Subfiles}{Info
  about the subfiles package can be found here}


Citations work the same, usign the master bib file: 
You need to remember to run bibtex on sub-file.aux.  \cite{ndmg}

\onlyinsubfile{%% 
  \bibliographystyle{IEEEtran}
  \begin{spacing}{0.5}
  {\footnotesize	\bibliography{example}}
  \end{spacing}
}
\end{document}
